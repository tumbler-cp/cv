\documentclass[a4paper,10pt]{article}
\usepackage[english,russian]{babel}
\usepackage[utf8]{inputenc}
\usepackage[T2A]{fontenc}
\usepackage[left=2cm,right=2cm,top=2cm,bottom=2cm]{geometry}
\usepackage{fontawesome5}
\pagenumbering{gobble}
\usepackage[
colorlinks,
urlcolor=red
]{hyperref}

\begin{document}
\begin{figure}[!htb]
    \begin{minipage}{.5\textwidth}
        \begin{flushleft}
            \begin{Large}
            Ходжаев Абдужалол Абдужаборович\\
            \end{Large}
            \begin{large}
                \textbf{Fullstack-разработчик}
            \end{large}
        \end{flushleft}
    \end{minipage}
    \begin{minipage}{.5\textwidth}
        \begin{flushright}
            \begin{small}
                \href{https://github.com/tumbler-cp}{tumbler-cp} \faGithub\\
                \href{https://t.me/arahnovuk}{arahnovuk} \faTelegram\\
                abdujalol04@mail.ru \faAt\\
                г.Санкт-Петербург, Россия \faMapMarker\\
                +7-951-662-49-18 \faPhone
            \end{small}
        \end{flushright}
    \end{minipage}
\end{figure}
\noindent 
Студент 3-го курса 
ИТМО/ПИиКТ,
активно развивающий навыки в веб-разработке,
Стремлюсь применить теоретические знания на практике.
Быстро осваиваю новые технологии и инструменты.
Внимателен к деталям и открыт к постоянной связи. Ищу возможность развиваться в направлении\\
\subsection*{Образование}
\hrule
\vspace*{5mm}
\begin{tabular}{c | p{0.8\textwidth}}
    2022 - 2026 & \textbf{Университет ИТМО.} Бакавлариат "Системное и прикладное программное обеспечение". Факультет программной инженерии и компьютерной техники (ФПИиКТ)  \\
\end{tabular}
\subsection*{Навыки}
\hrule
\vspace*{5mm}
\begin{tabular}{r | l }
    \textbf{Языки программирования} & Java, Python, C++, JavaScript, TypeScript\\
    \textbf{Фреймворки и библиотеки} & Spring Framework, Django, Hibernate, React, Tailwind \\
    \textbf{Инструменты} & Git, Maven, Gradle, Swagger, Docker, Bash, Linux, npm, yarn, Vite \\
    \textbf{Базы данных} & PostgrSQL, Redis \\
    \textbf{Тестирование} & Junit5, JMeter, Cypress, Selenium, Postman \\
\end{tabular}
\subsection*{Проекты}
\hrule
\vspace*{5mm}
\textbf{DroneDeliverManager} Декабрь 2024 - Январь 2025\\
\faGithub \hspace*{3mm} \textbf{\url{https://github.com/tumbler-cp/CWBack}} \\
\faGithub \hspace*{3mm} \textbf{\url{https://github.com/tumbler-cp/DroneNotificationReact}} \\
Система управление доставками дронами, а также склад товаров. \\
\textbf{Spring Framework, Docker, PostgreSQL, React, Tailwind} \\

\vspace*{5mm}
\noindent\textbf{WorkerManager} Ноябрь 2024\\
\faGithub \hspace*{3mm} \textbf{\url{https://github.com/tumbler-cp/WorkerManagerWithoutBoot}} \\
\faGithub \hspace*{3mm} \textbf{\url{https://github.com/tumbler-cp/workersManagerReact}} \\
Система управление доставками дронами, а также склад товаров. \\
\textbf{Spring Framework, Docker, PostgreSQL, React, Tailwind} \\

\subsection*{Дополнительно}
\hrule
\vspace*{5mm}
\subsubsection*{Языки}
\textbf{Русский} - Уровень компетентного владения \\
\textbf{Английский} - Advanced (C1) \\
\textbf{Таджикский} - Носитель
\subsubsection*{Soft Skills}
\begin{itemize}
    \item Адаптивность
    \item Коммуникация
    \item Публичные выступления
\end{itemize}
\subsubsection*{Хобби}
Embedded разработка, Разработка игр, Системная разработка
\end{document}